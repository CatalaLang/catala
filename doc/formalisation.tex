\documentclass[11pt,a4paper]{article}

\usepackage[T1]{fontenc}
\usepackage[utf8]{inputenc}
\usepackage[english]{babel}
\usepackage{fullpage}
\usepackage{lmodern}
\usepackage{amsmath,amssymb}
\usepackage{mathpartir}
\usepackage{xcolor}
\usepackage{booktabs}
\usepackage{nicefrac}
\usepackage{csquotes}

\newtheorem{theorem}{Theorem}
\newtheorem{lemma}{Lemma}

%% Syntax
\newcommand{\synvar}[1]{\ensuremath{#1}}
\newcommand{\synkeyword}[1]{\textcolor{red!60!black}{\texttt{#1}}}
\newcommand{\synpunct}[1]{\textcolor{black!40!white}{\texttt{#1}}}
\newcommand{\synname}[1]{\ensuremath{\mathsf{#1}}}

\newcommand{\synbool}{\synkeyword{bool}}
\newcommand{\synnum}{\synkeyword{num}}
\newcommand{\syndate}{\synkeyword{date}}
\newcommand{\synvec}{\synkeyword{vec~}}
\newcommand{\synopt}{\synkeyword{opt~}}

\newcommand{\syncase}{\synkeyword{case~}}
\newcommand{\synof}{\synkeyword{~of~}}
\newcommand{\synif}{\synkeyword{if~}}
\newcommand{\synthen}{\synkeyword{~then~}}
\newcommand{\synelse}{\synkeyword{~else~}}
\newcommand{\synlet}{\synkeyword{let~}}
\newcommand{\synin}{\synkeyword{~in~}}
\newcommand{\synequal}{\synpunct{~=~}}
\newcommand{\synjust}{~\synpunct{:\raisebox{-0.9pt}{-}}~}
\newcommand{\syntyped}{~\synpunct{:}~}
\newcommand{\syndot}{\synpunct{.}~}
\newcommand{\syntrue}{\synkeyword{true}}
\newcommand{\synfalse}{\synkeyword{false}}
\newcommand{\synop}{\synpunct{\odot}}
\newcommand{\synlambda}{\synpunct{$\lambda$}~}
\newcommand{\synand}{\synpunct{\wedge}}
\newcommand{\synor}{\synpunct{\vee}}
\newcommand{\synlparen}{\synpunct{(}}
\newcommand{\synrparen}{\synpunct{)}}
\newcommand{\synlbracket}{\synpunct{\{}}
\newcommand{\synrbracket}{\synpunct{\}}}
\newcommand{\synlangle}{\synpunct{$\langle$}}
\newcommand{\synrangle}{\synpunct{$\rangle$}}
\newcommand{\synmid}{\synpunct{~$|$~}}
\newcommand{\synemptydefault}{\synvar{\varnothing}}
\newcommand{\synerror}{\synvar{\circledast}}
\newcommand{\synmerge}{\synkeyword{~++~}}

\newcommand{\synvardef}{\synkeyword{definition~}}
\newcommand{\synscopecall}{\synkeyword{scope\_call~}}

\newcommand{\synarrow}{~\synpunct{$\rightarrow$}~}
\newcommand{\syntypsum}{\synpunct{$+$}}
\newcommand{\syntypproduct}{\synpunct{$\times$}}
\newcommand{\syndiamond}{\;\synpunct{$\diamond$}\;}
\newcommand{\syncomma}{\synpunct{,}}
\newcommand{\synellipsis}{\synpunct{,$\ldots$,}}

\newcommand{\syndef}{$ ::= $}
\newcommand{\synalt}{\;$|$\;}

\newcommand{\synhole}{\synvar{\cdot}}

%% Typing
\newcommand{\typctx}[1]{\textcolor{orange!90!black}{\ensuremath{#1}}}
\newcommand{\typempty}{\typctx{\varnothing}}
\newcommand{\typcomma}{\typctx{,\;}}
\newcommand{\typvdash}{\typctx{\;\vdash\;}}
\newcommand{\typcolon}{\typctx{\;:\;}}
\newcommand{\typlpar}{\typctx{(}}
\newcommand{\typrpar}{\typctx{)}}

%% Well-formedness
\newcommand{\wfctx}[1]{\textcolor{green!50!black}{\ensuremath{#1}}}
\newcommand{\wfkeyword}[1]{\textcolor{green!50!black}{\texttt{#1}}}

\newcommand{\wfgammaempty}{\wfctx{\varnothing_\Gamma}}
\newcommand{\wfdeltaempty}{\wfctx{\varnothing_\Delta}}
\newcommand{\wfcomma}{\wfctx{,}\;}
\newcommand{\wfok}{\wfkeyword{~ok}}
\newcommand{\wfokin}{\wfkeyword{~ok~in~}}
\newcommand{\wfArrow}{\wfctx{~\Rightarrow~}}
\newcommand{\wfarrow}{\wfctx{~\rightarrow~}}
\newcommand{\wfeq}{\;\wfctx{=}\;}
\newcommand{\wfvdash}{\;\wfctx{\vdash}\;}
\newcommand{\wfsemicolon}{\wfctx{;\;}}

%% Evaluation and execution
\newcommand{\exctx}[1]{\textcolor{blue!80!black}{\ensuremath{#1}}}

\newcommand{\Omegaarg}{\Omega_{arg}}
\newcommand{\excaller}{\exctx{\complement}}
\newcommand{\excomma}{\exctx{,}\;}
\newcommand{\exvdash}{\;\exctx{\vdash}\;}
\newcommand{\exempty}{\exctx{\varnothing}}
\newcommand{\exemptyv}{\exctx{\varnothing_v}}
\newcommand{\exemptyarg}{\exctx{\varnothing_{arg}}}
\newcommand{\exvarmap}{\exctx{~\mapsto~}}
\newcommand{\exscopemap}{\exctx{~\rightarrowtail~}}
\newcommand{\exArrow}{\exctx{~\Rrightarrow~}}
\newcommand{\exeq}{\exctx{\;=\;}}
\newcommand{\exeval}{\exctx{\;\longrightarrow\;}}
\newcommand{\exevalstar}{\exctx{\;\longrightarrow^*\;}}
\newcommand{\exat}{\exctx{\texttt{\;@\;}}}
\newcommand{\exsemicolon}{\exctx{;~}}
\newcommand{\excomp}{\dashrightarrow}


\title{Formalization of the Catala language}
\date{}
\author{Nicolas Chataing, Denis Merigoux}

\begin{document}
\maketitle

\section{Default calculus}

We chose to present the core Catala as a lambda-calculus augmented by a special 
\enquote{default} expression. This special expression enables to deal with 
the logical structure underlying tax law.

\subsection{Syntax}
\label{sec:defaultcalc:syntax}

\begin{center}
\begin{tabular}{lrrll}
  Type&\synvar{\tau}&\syndef&\synbool&boolean type\\
  &&\synalt&\synvar{\tau}\synarrow\synvar{\tau}&function type \\
  &&&&\\
  Expression&\synvar{e}&\syndef&\synvar{x}\synalt\syntrue\synalt\synfalse&variable, literal\\
  &&\synalt&\synlambda\synlparen\synvar{x}\syntyped\synvar{\tau}\synrparen\syndot\synvar{e}\synalt\synvar{e}\;\synvar{e}&$\lambda$-calculus\\
  &&\synalt&\synvar{d}&default term\\
  &&&&\\
  Default&\synvar{d}&\syndef&\synlangle\synvar{e}\synjust\synvar{e}\synmid $[\synvar{d}^*]$\synrangle&default term\\
  &&\synalt&\synerror&conflict error term\\
  &&\synalt&\synemptydefault&empty error term\\
\end{tabular}
\end{center}

The syntax of the lambda calculus is enriched with a construction coming from 
default logic. Particularly, we focus on a subset of default logic called 
categorical, prioritized default logic. In this setting, a default is a logical 
rule of the form $A \synjust B$ where $A$ is the justification of the rule and 
$B$ is the consequence. The rule can only be applied if $A$ is consistent with 
our current knowledge. If multiple rules $A \synjust B_1$ and $A \synjust B_2$
can be applied at the same time, then only one of them is applied through 
an explicit order of the rules.

To translate this form of logic inside our programming language, we set $A$ to 
be an expression that can be evaluated to \syntrue{} or \synfalse{}, and $B$
the value that the default should evaluate to if $A$ is true. If $A$ is false,
then we look up for other rules of lesser priority to apply. This priority 
is encoded trough a syntactic tree data structure\footnote{Thanks to Pierre-Évariste Dagand for this insight.}. 

In the term \synlangle\synvar{e_{\text{just}}}\synjust
\synvar{e_{\text{cons}}}\synmid $[\synvar{d}^*]$\synrangle, \synvar{e_{\text{just}}} 
is the justification $A$, \synvar{e_{\text{cons}}} is the consequence $B$ and 
$[\synvar{d}^*]$ is the list of rules to be considered if \synvar{e_{\text{just}}} 
evaluates to \synfalse{}. 
 
Of course, this evaluation scheme can fail if no more 
rules can be applied, or if two or more rules of the same priority have their 
justification evaluate to \syntrue{}. The error terms \synerror{} and \synemptydefault{}
encode these failure cases.

\subsection{Typing}
\label{sec:defaultcalc:typing}

The typing judgment \fbox{$\typctx{\Gamma}\typvdash\synvar{e}\typcolon\synvar{\tau}$} reads as
\enquote{under context $\typctx{\Gamma}$, expression $\synvar{e}$ has type $\synvar{\tau}$}.

\begin{center}
  \begin{tabular}{lrrll}
    Typing context&\typctx{\Gamma}&\syndef&\typempty&empty context\\
    (unordered map)&&\synalt&\typctx{\Gamma}\typcomma\synvar{x}\typcolon\synvar{\tau}&typed variable\\
  \end{tabular}
\end{center}


We start by the usual rules of simply-typed lambda calculus.
\begin{center}
\begin{mathpar}
  \inferrule[Var]{ }{
      \typctx{\Gamma}\typcomma\synvar{x}\typcolon\synvar{\tau}\typvdash\synvar{x}\syntyped\synvar{\tau}
  }

  \inferrule[Abs]
  {\typctx{\Gamma}\typcomma\synvar{x}\typcolon\synvar{\tau}\typvdash\synvar{e}\typcolon\synvar{\tau'}}
  {\typctx{\Gamma}\typvdash\synlambda\synlparen\synvar{x}\syntyped{\tau}\synrparen\syndot\synvar{e}\typcolon\synvar{\tau}\synarrow\synvar{\tau'}}

  \inferrule[App]
  {
    \typctx{\Gamma}\typvdash\synvar{e_1}\typcolon\synvar{\tau_2}\synarrow\synvar{\tau_1}\\
    \typctx{\Gamma}\typvdash\synvar{e_2}\typcolon\synvar{\tau_2}
  }
  {\typctx{\Gamma}\typvdash\synvar{e_1}\;\synvar{e_2}\typcolon\synvar{\tau_1}}
\end{mathpar}
\end{center}  

Then we move to the special default terms. First, the error terms that stand for 
any type.

\begin{center}
\begin{mathpar}
  \inferrule[ConflictError]{}{\typctx{\Gamma}\typvdash\synerror\typcolon\synvar{\tau}}

  \inferrule[EmptyError]{}{\typctx{\Gamma}\typvdash\synemptydefault\typcolon\synvar{\tau}}
\end{mathpar}
\end{center}

Now the interesting part for the default terms. As mentioned earlier, the 
justification \synvar{e_{\text{just}}} is a boolean, while \synvar{e_{\text{cons}}}
can evaluate to any value. \TirName{DefaultBase} specifies how the tree structure 
of the default should be typed.

\begin{center}
\begin{mathpar}
  \inferrule[DefaultBase]
  {
    \typctx{\Gamma}\typvdash\synvar{e_{\text{just}}}\typcolon\synbool\\
    \typctx{\Gamma}\typvdash\synvar{e_{\text{cons}}}\typcolon\synvar{\tau}\\
    \forall i\in[\![1;n]\!],\;\typctx{\Gamma}\typvdash\synvar{d_i}\typcolon{\tau}
  }
  {\typctx{\Gamma}\typvdash\synlangle\synvar{e_{\text{just}}}\synjust\synvar{e_{\text{cons}}}\synmid 
  \synvar{d_1}\synellipsis\synvar{d_n}\synrangle\typcolon\synvar{\tau}}
\end{mathpar}
\end{center}
The situation becomes more complex in the presence of functions. Indeed, want 
our default expressions to depend on parameters. By only allowing \synvar{e_{\text{just}}}
to be \synbool{}, we force the user to declare the parameters in a \synlambda 
that wraps the default from the outside. Using this scheme, all the expressions 
inside the tree structure of the default will depend on the same bound variable 
\synvar{x}.

However, this scheme is not adapted to the way Catala will expose default 
to the user. Indeed, each legislative article will yield a rule that will be 
encoded into a \synvar{e_{\text{just}}}\synjust\synvar{e_{\text{cons}}} expression.
Later, all these expression will be collected and turned into the tree data 
structure according to the priorities defined by the law. But if each 
\synvar{e_{\text{just}}}\synjust\synvar{e_{\text{cons}}} expression depends on 
a parameter, the source code will include bindings for free variables in each 
of the default tree node. Rather than to hoist these bindings outside of the 
default and perform $\alpha$-renaming, we prefer to explicitly allow bindings 
inside the default tree nodes. Hence, we include a dedicated typing rule, 
\TirName{DefaultFun}, as well as a later rule for performing $\beta$-reduction 
under the default.

\begin{center}
\begin{mathpar}
  \inferrule[DefaultFun]
  {
    \typctx{\Gamma}\typcomma\synvar{x}\typcolon\synvar{\tau}\typvdash\typvdash\
    \synlangle\synvar{e_{\text{just}}}\synjust\synvar{e_{\text{cons}}}\synmid\synrangle
    \typcolon\synvar{\tau'}\\
    \forall i\in[\![1;n]\!],\;
    \typctx{\Gamma}\typvdash\synvar{d_i}\typcolon\synvar{\tau}\synarrow\synvar{\tau'}
  }
  {\typctx{\Gamma}\typvdash
  \synlangle\synlambda\synlparen\synvar{x}\syntyped\synvar{\tau}\synrparen\syndot\synvar{e_{\text{just}}}\synjust
  \synlambda\synlparen\synvar{x}\syntyped\synvar{\tau}\synrparen\syndot\synvar{e_{\text{cons}}}\synmid
  \synvar{d_1}\synellipsis\synvar{d_n}\synrangle\typcolon\tau\synarrow\tau'}
\end{mathpar}
\end{center}

\subsection{Evaluation}

We give this default calculus small-step, structured operational semantics. The 
one-step reduction judgment is of the form \fbox{\synvar{e}\exeval\synvar{e'}}.

In our simple language, values are just booleans or functions and we use a evaluation contexts. 
Evaluation contexts are contexts where the hole can only occur at some
admissible positions that often described by a grammar.

\begin{center}
  \begin{tabular}{lrrll}
    Values&\synvar{v}&\syndef&\synlambda\synlparen\synvar{x}\syntyped\synvar{\tau}\synrparen\syndot\synvar{e}&functions\\
                    &&\synalt&\syntrue\synalt\synfalse & booleans\\
    Evaluation &\synvar{C}&\syndef&\synhole\;\synvar{e}\synalt\synvar{v}\;\synhole&function application\\
       contexts&&\synalt&\synlangle\synhole\synjust\synvar{e}\synmid \synvar{d_1}\synellipsis\synvar{d_n}\synrangle&default justification evaluation\\
                    &&\synalt&\synlangle\syntrue\synjust\synhole\synmid \synvar{d_1}\synellipsis\synvar{d_n}\synrangle&default consequence evaluation\\
                    &&\synalt&\synlangle\synfalse\synjust\synvar{e}\synmid \synvar{v_1}\synellipsis\synvar{v_{i-1}}
                    \syncomma\synhole\syncomma\synvar{d_{i+1}}\synellipsis\synvar{d_n}\synrangle&sub-default evaluation
  \end{tabular}
\end{center}

Note that our language is call-by-value.
First, we present the usual reduction rules for beta-reduction 
(function application and default application) 
and evaluation inside a context hole.

\begin{center}
\begin{mathpar}
   \inferrule[Context]
  {\synvar{e}\exeval\synvar{e'}}
  {\synvar{C}[\synvar{e}]\exeval\synvar{C}[\synvar{e'}]}

 \inferrule[$\beta_v$]{}{
   (\synlambda\synlparen\synvar{x}\syntyped\synvar{\tau}\synrparen\syndot{e})\;\synvar{v}
    \exeval\synvar{e}[\synvar{x}\mapsto\synvar{v}]
 }

   \inferrule[$\beta_d$]{}{
     \synlangle \synlambda\synlparen\synvar{x}\syntyped\synvar{\tau}\synrparen\syndot\synvar{e_{\text{just}}}\synjust
     \synlambda\synlparen\synvar{x}\syntyped\synvar{\tau}\synrparen\syndot\synvar{e_{\text{cons}}}\synmid 
     \synvar{d_1}\synellipsis\synvar{d_n}\synrangle\;\synvar{v}\exeval\\
      \synlangle \synvar{e_{\text{just}}}[\synvar{x}\mapsto\synvar{v}]\synjust
     \synvar{e_{\text{cons}}}[\synvar{x}\mapsto\synvar{v}]\synmid 
     \synvar{d_1}\;\synvar{v}\synellipsis\synvar{d_n}\;\synvar{v}\synrangle
     }


\end{mathpar}
\end{center}

Now we have to describe how the default terms reduce. Thanks to a the 
\TirName{Context} rule, we can suppose that the justification of the default 
is already reduced to a variable \synvar{v}. By applying the beta-reduction 
rule $\beta_d$, we can further reduce to the case where \synvar{v} is a boolean.
This is where we encode our default logic evaluation semantics. If \synvar{v} is 
\syntrue{}, then this rule applies and we reduce to the consequence. We don't 
even have to consider rules of lower priority lower in the tree.

\begin{center}
\begin{mathpar}
  \inferrule[DefaultJustifTrue]
  {}
  {\synlangle \syntrue\synjust \synvar{e}\synmid \synvar{d_1}\synellipsis\synvar{d_n}\synrangle\exeval e}
\end{mathpar}
\end{center}

\newcommand{\exDeval}{\exctx{\;\Downarrow\;}}
\newcommand{\exnotDeval}{\exctx{\;\nRightarrow\;}}

$$ A([D_1, \ldots, D_n]) = \#\{i | D_i \exnotDeval \synemptydefault \} $$
$$ B([D_1, \ldots, D_n]) = e~|~D_i \exDeval e \wedge e \neq \synemptydefault $$

\begin{center}
\begin{mathpar}
  \inferrule[NoJustifTrue]
  {\synvar{b}\exevalstar\bot\\A([D_1, \ldots, D_n]) = 0}
  {\{\nicefrac{\synvar{b}}{\synvar{e}},[\synvar{D_1}, \ldots, \synvar{D_n}]\}\exDeval\synemptydefault}

  \inferrule[DefaultOk]
  {\synvar{b}\exevalstar\bot\\A([D_1, \ldots, D_n]) = 1\\B([D_1, \ldots, D_n]) = \synvar{e}}
  {\{\nicefrac{\synvar{b}}{\synvar{e}},[\synvar{D_1}, \ldots, \synvar{D_n}]\}\exDeval\synvar{e}}
  
  \inferrule[DefaultConflict]
  {\synvar{b}\exevalstar\bot\\A([D_1, \ldots, D_n]) > 1}
  {\{\nicefrac{\synvar{b}}{\synvar{e}},[\synvar{D_1}, \ldots, \synvar{D_n}]\}\exDeval\synerror}

\end{mathpar}
\end{center}



\begin{center}
\begin{mathpar}


  \inferrule[Default]
  {\synvar{D}\exDeval\synvar{e}}
  {\synvar{D}\exeval\synvar{e}}
  
  \inferrule[ConflictError]
  {e\exeval\synerror}
  {C[e]\exeval\synerror}

  \inferrule[EmptyError]
  {e\exeval\synemptydefault}
  {C[e]\exeval\synemptydefault}
\end{mathpar}
\end{center}

\end{document}
